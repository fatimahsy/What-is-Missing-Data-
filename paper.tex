% Options for packages loaded elsewhere
\PassOptionsToPackage{unicode}{hyperref}
\PassOptionsToPackage{hyphens}{url}
\PassOptionsToPackage{dvipsnames,svgnames,x11names}{xcolor}
%
\documentclass[
  letterpaper,
  DIV=11,
  numbers=noendperiod]{scrartcl}

\usepackage{amsmath,amssymb}
\usepackage{iftex}
\ifPDFTeX
  \usepackage[T1]{fontenc}
  \usepackage[utf8]{inputenc}
  \usepackage{textcomp} % provide euro and other symbols
\else % if luatex or xetex
  \usepackage{unicode-math}
  \defaultfontfeatures{Scale=MatchLowercase}
  \defaultfontfeatures[\rmfamily]{Ligatures=TeX,Scale=1}
\fi
\usepackage{lmodern}
\ifPDFTeX\else  
    % xetex/luatex font selection
\fi
% Use upquote if available, for straight quotes in verbatim environments
\IfFileExists{upquote.sty}{\usepackage{upquote}}{}
\IfFileExists{microtype.sty}{% use microtype if available
  \usepackage[]{microtype}
  \UseMicrotypeSet[protrusion]{basicmath} % disable protrusion for tt fonts
}{}
\makeatletter
\@ifundefined{KOMAClassName}{% if non-KOMA class
  \IfFileExists{parskip.sty}{%
    \usepackage{parskip}
  }{% else
    \setlength{\parindent}{0pt}
    \setlength{\parskip}{6pt plus 2pt minus 1pt}}
}{% if KOMA class
  \KOMAoptions{parskip=half}}
\makeatother
\usepackage{xcolor}
\setlength{\emergencystretch}{3em} % prevent overfull lines
\setcounter{secnumdepth}{5}
% Make \paragraph and \subparagraph free-standing
\ifx\paragraph\undefined\else
  \let\oldparagraph\paragraph
  \renewcommand{\paragraph}[1]{\oldparagraph{#1}\mbox{}}
\fi
\ifx\subparagraph\undefined\else
  \let\oldsubparagraph\subparagraph
  \renewcommand{\subparagraph}[1]{\oldsubparagraph{#1}\mbox{}}
\fi


\providecommand{\tightlist}{%
  \setlength{\itemsep}{0pt}\setlength{\parskip}{0pt}}\usepackage{longtable,booktabs,array}
\usepackage{calc} % for calculating minipage widths
% Correct order of tables after \paragraph or \subparagraph
\usepackage{etoolbox}
\makeatletter
\patchcmd\longtable{\par}{\if@noskipsec\mbox{}\fi\par}{}{}
\makeatother
% Allow footnotes in longtable head/foot
\IfFileExists{footnotehyper.sty}{\usepackage{footnotehyper}}{\usepackage{footnote}}
\makesavenoteenv{longtable}
\usepackage{graphicx}
\makeatletter
\def\maxwidth{\ifdim\Gin@nat@width>\linewidth\linewidth\else\Gin@nat@width\fi}
\def\maxheight{\ifdim\Gin@nat@height>\textheight\textheight\else\Gin@nat@height\fi}
\makeatother
% Scale images if necessary, so that they will not overflow the page
% margins by default, and it is still possible to overwrite the defaults
% using explicit options in \includegraphics[width, height, ...]{}
\setkeys{Gin}{width=\maxwidth,height=\maxheight,keepaspectratio}
% Set default figure placement to htbp
\makeatletter
\def\fps@figure{htbp}
\makeatother

\KOMAoption{captions}{tableheading}
\makeatletter
\@ifpackageloaded{caption}{}{\usepackage{caption}}
\AtBeginDocument{%
\ifdefined\contentsname
  \renewcommand*\contentsname{Table of contents}
\else
  \newcommand\contentsname{Table of contents}
\fi
\ifdefined\listfigurename
  \renewcommand*\listfigurename{List of Figures}
\else
  \newcommand\listfigurename{List of Figures}
\fi
\ifdefined\listtablename
  \renewcommand*\listtablename{List of Tables}
\else
  \newcommand\listtablename{List of Tables}
\fi
\ifdefined\figurename
  \renewcommand*\figurename{Figure}
\else
  \newcommand\figurename{Figure}
\fi
\ifdefined\tablename
  \renewcommand*\tablename{Table}
\else
  \newcommand\tablename{Table}
\fi
}
\@ifpackageloaded{float}{}{\usepackage{float}}
\floatstyle{ruled}
\@ifundefined{c@chapter}{\newfloat{codelisting}{h}{lop}}{\newfloat{codelisting}{h}{lop}[chapter]}
\floatname{codelisting}{Listing}
\newcommand*\listoflistings{\listof{codelisting}{List of Listings}}
\makeatother
\makeatletter
\makeatother
\makeatletter
\@ifpackageloaded{caption}{}{\usepackage{caption}}
\@ifpackageloaded{subcaption}{}{\usepackage{subcaption}}
\makeatother
\ifLuaTeX
  \usepackage{selnolig}  % disable illegal ligatures
\fi
\usepackage{bookmark}

\IfFileExists{xurl.sty}{\usepackage{xurl}}{} % add URL line breaks if available
\urlstyle{same} % disable monospaced font for URLs
\hypersetup{
  pdftitle={What is Missing Data?},
  pdfauthor={Fatimah Yunusa},
  colorlinks=true,
  linkcolor={blue},
  filecolor={Maroon},
  citecolor={Blue},
  urlcolor={Blue},
  pdfcreator={LaTeX via pandoc}}

\title{What is Missing
Data?\thanks{https://github.com/fatimahsy/What-is-Missing-Data-.git}}
\usepackage{etoolbox}
\makeatletter
\providecommand{\subtitle}[1]{% add subtitle to \maketitle
  \apptocmd{\@title}{\par {\large #1 \par}}{}{}
}
\makeatother
\subtitle{What should we do about it?}
\author{Fatimah Yunusa}
\date{March 5, 2024}

\begin{document}
\maketitle

\section{Introduction}\label{introduction}

In the data acquisition process, no matter how efficient and careful we
are, there is always a high possibility that we will have missing
data.Missing data poses a challenge when cunducting statistical analysis
because it often impacts reliability and validity. This Mini-essay
delves into the nature of missing data, its importance, the types,
causes of missing data, consequences of missing data and how to properly
handle missing data.

\subsection{What is missing data?}\label{what-is-missing-data}

Missing data is when there are missing observations in the data set. It
is variables that were not obtained for different observations.

\subsection{Why is it important to us?}\label{why-is-it-important-to-us}

Missing data is important to us because it adds further uncertainty to
our statistical analysis. We must think about the possibility of missing
data because it also enables us to think about other missing variables
that we have not accounted for or provided measures of.

\section{Types of Missing Data}\label{types-of-missing-data}

When we identify missing data, the first thing we ough to do is try and
figure out what type of missing data it is. Identifying the type enables
us to understand how to deal with it. There are three main
classifications of missing data:

-Missing Completely At Random -Missing At Random -Missing Not At Random

Data are Missing Completely At Random when observations do not show up
in the data set across all observations.This is a case that rarely
happens but in cases like that,there is less concern relating to summary
statistics or inference.

When observations are Missing at Random, they are missing from the data
set in relation to the other variables that are included in the data
set. This case accounts for data that has some sort of response bias
embedded into it.

When observations are Missing Not At Random, this means that the
probability of missing data is connected to the values that have not
been observed. So the missing data has a relation to the variable or
variables being measured. For example if in a medical study, patients
with severe side effects are less likely to return for follow-up visits,
the missing data on side effects is not random but related to the
severity of those side effects.

\section{What Causes Missing Data?}\label{what-causes-missing-data}

Data might be missing for many reasons. Data might be missing because of
a lack of responses from participants, errors in data entry, failure of
research equipment, flaws in the design of the study, deliberate
omission of data from respondents, data processing errors and many other
unprecedented events.

\section{Consequences of ignoring Missing
Data}\label{consequences-of-ignoring-missing-data}

Missing data can have several consequences and this is why it is
important to identify and deal with them.They can introduce bias to the
results, increase variability, lead to wrong inferences, reduce external
validity and lead to incorrect conclusions.

\section{Handling Missing Data}\label{handling-missing-data}

Dealing with missing data requires careful consideration fo many
factors.There are many tactics we can use to deal with these missing
variables. These include:

\begin{itemize}
\tightlist
\item
  dropping the observation that has missing values
\item
  using multiple imputation
\item
  General data imputation
\item
  Creating a dummy variable
\item
  Using predictive Models
\item
  Assigning weights to different observations
\end{itemize}

\section{Conclusion}\label{conclusion}

In conclusion, missing data plays a big role in our statistical
analysis. It can potentially lead us to presenting biased results and
misleading conclusions. It is important that the researchers understand
the different types of missing data and how to handle them. They must
decide which method of handling is most appropriate for their data type.



\end{document}
